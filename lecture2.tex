\chapter*{Lecture 2}


\section*{Linear Systems}

Linear systems, also called filtering, has its output calculated as convolution of the input signal and 
the impulse response of the system(filter) in time domain, as shown in equation \ref{convolution}.

\begin{equation}
y(t) = \int_{-\infty}^{\infty}h(\tau)x(t-\tau)d\tau = x(t) \ast h(t)
\label{convolution}
\end{equation}
where $x(t)$ is the input signal and $h(t)$ is the impulse response of the linear system. 
Equation \ref{convolution} implies that:

\begin{itemize}
\item[(1)] if $x(t) = \delta(t)$, then $y(t) = h(t)$. Because
\begin{equation}
\int_{-\infty}^{\infty}h(\tau)\delta (t-\tau) d\tau = \int_{-\infty}^{\infty}h(t-\lambda)\delta (\lambda) d\lambda = h(t)
\end{equation}
\item[(2)] Linearity. $k x(t) \implies k y(t)$, $x_1(t) + x_2(t) \implies y_1(t) + y_2(t)$.
\item[(3)] Time invariant. $x(t-\Delta) \implies y(t-\Delta)$, i.e. $h(\tau)$ does not depend on $t$ in equation \ref{convolution}.
\item[(4)] Causality. $h(\tau) = 0$ for $\tau < 0$. To make the filter physically realizable, this constraint must be fulfilled. 
\end{itemize}

The Fourier transform of the convolution equation, \ref{convolution} will be 
\begin{equation}
Y(\omega) = H(\omega)X(\omega)
\end{equation}
where $H(\omega) = \mathbb{FT}[h(t)]$ is called the frequency function of the filter(linear system). 

For example, 
\begin{equation}
\begin{aligned}
& x(t)=e^{j\omega_0t}   \implies {X}(\omega) = 2\pi\delta(\omega-\omega_0) \\
& {Y}(\omega) = {H}(\omega) 2 \pi \delta (\omega - \omega_0)  \\
&  y(t) = \frac{1}{2\pi} \int_{-\infty}^{\infty} {H}(\omega)2\pi\delta(\omega - \omega_0) e^{j\omega t} d\omega 
= {H}(\omega_0)e^{j\omega_0 t}
\end{aligned}
\end{equation}

For every frequency function of linear system, $|{H}(\omega)| = A(\omega)$ is called the amplitude function, 
and $\angle{H}(\omega) = \Phi(\omega)$ is called the phase function of the linear system. 
Thus ${H}(\omega_0)$ can be written as $A(\omega_0)e^{j\Phi(\omega_0)}$, 
and the output will be $A(\omega_0)e^{j\omega_0 t + j\Phi(\omega_0)}$.

\section*{Sampled Data System and Discrete-Time Systems}

Discrete time signal $\{x(n)\}$, $n=0 \pm 1 \pm 2 \dots$ is often associated with a continuous time signal $x(t)$ by sampling.
i.e. $x(n) \triangleq x(n T_s)$, where $T_s$ is the sampling period. $f_s = 1/T_s$ is the sampling
frequency in Hz. $\omega_s = 2\pi f_s = 2\pi / T_s$ is the sampling angular frequency in radians per second. 
$f_s/2$ is called Nyquist frequency. Please note the difference between Nyquist frequency and Nyquist rate.

\begin{equation}
{X}(\omega) \triangleq \sum_{n = -\infty}^{\infty}x(n)e^{-j\omega T_s n}
\label{DTFT}
\end{equation} 

The Discrete-Time Fourier Transform of a sampled sequence is defined in equation \ref{DTFT},
where $\omega T_s$ has the unit of radians per sample. It can also be regarded as the normalized frequency because of the 
other form $\omega/f_s$. We can further expand the $\omega$ into $2\pi f$, then we have $\omega T_s = 2\pi f /f_s$. 
Note that ${X}{\omega} = {X}(\omega + k\omega_s)$, $k = 0 \pm 1 \pm 2 \dots$, which can be proved 
by the periodicity of the complex exponential function. The periodicity of the frequency function and the sampling frequency 
also explain the aliasing effects during sampling of real-numbered signals. 

The inverse DTFT is defined as 
\begin{equation}
x(n) = \frac{1}{2\pi f_s} \int_0^{2\pi f_s} {X}(\omega)e^{j\omega T_s n} d\omega
\end{equation} 

Take unit delay for instance, we have $y(n) \triangleq x(n-1)$, then
\begin{equation}
{Y}(\omega) = \sum_{-\infty}^{\infty}x(n-1)e^{-j\omega T_s n} = 
\sum_{m = -\infty}^{\infty}x(m)e^{-j\omega T_s m} e^{-j\omega T_s}= e^{-j\omega T_s} {X}(\omega)
\label{eq:ftdelay}
\end{equation}

The general difference equation of a linear system can be written as 
\begin{equation}
y(n) = - \sum_{k=1}^{N_a} a_k y(n-k) + \sum_{k=0}^{N_b} b_k x(n-k)
\label{eq:lineardiff}
\end{equation}

By the result of example \ref{eq:ftdelay}, we take the Fourier transform of equation \ref{eq:lineardiff}
to obtain the transfer function of the linear system, described in equation \ref{eq:transferomega}. 
\begin{equation}
Y(\omega) = \frac{\sum_{k=0}^{N_b}b_k e^{-j\omega T_s k}}{\sum_{k=1}^{N_a} a_k e^{-j\omega T_s k} + 1} X(\omega)
\label{eq:transferomega}
\end{equation}

If $z = e^{j\omega T_s}$, then we can change the transfer function \ref{eq:transferomega} of $\omega$ into
Z transformation. 
\begin{equation}
H(z) = \frac{\sum_{k=0}^{N_b}b_k z^{-k}}{1+\sum_{k=1}^{N_a}a_k z^{-k}} = \frac{b(z)}{a(z)}
\label{eq:transferz}
\end{equation} 

Roots of $a(z)$ are called poles and roots of $b(k)$ are called zeros. The linear system is 
stable if all poles are inside the unit circle. We can expend the close form of the impulse 
response in equation \ref{eq:transferomega} and \ref{eq:transferz} into series, 
as shown in equation \ref{eq:expand}.
\begin{equation}
\begin{aligned}
& H(\omega)=\frac{\sum_{k=0}^{N_b}b_k e^{-j\omega T_s k}}{\sum_{k=1}^{N_a} a_k e^{-j\omega T_s k} + 1}
=\sum_{k=0}^{\infty}h(k) e ^{-j\omega T_s k} \\
& H(z) = \frac{\sum_{k=0}^{N_b}b_k z^{-k}}{1+\sum_{k=1}^{N_a}a_k z^{-k}}  
=\sum_{k=0}^{\infty}h(k) z^{-k} \\
\end{aligned}
\label{eq:expand}
\end{equation} 

where $h(k)$ is the impulse response of the system in time doamin. The system output will be
\begin{equation}
Y(\omega) = H(\omega)X(\omega) = \sum_{k=0}^{\infty}h(k)e^{-j\omega T_s k} X(\omega)
\end{equation}

After inverse DTFT, we have
\begin{equation}
\begin{aligned}
& y(n)  = \frac{T_s}{2\pi}\int_{0}^{2\pi /T_s} \sum_{k=0}^{\infty}h(k)e^{-j\omega T_s k}X(\omega) e^{j\omega T_s n} d\omega \\
      & = \sum_{k=0}^{\infty} h(k) \frac{T_s}{2\pi}\int_{0}^{2\pi /T_s} X(\omega) e^{j\omega T_s (n-k)} d\omega\\
     & = \sum_{k=0}^{\infty} h(k) x(n-k) \\ 
\end{aligned}
\end{equation}
This is the convolution(filtering) of the input signal and the linear system. 

\section*{Sampling Process}

The mathematical model for sampling can be derived by multiplication of the continuous 
input signal $x(t)$ by Dirac Delta pulse train $\delta_T(t) = \sum_{n=-\infty}^{\infty}\delta(t-nT_s)$.
Thus there will be equation \ref{eq:sample}

\begin{equation}
\begin{aligned}
& x_c(t)  = \sum_{n=-\infty}^{\infty}\delta (t-nT_s)x(t) \\
         & = \sum_{n=-\infty}^{\infty}\delta (t-nT_s)x(nT_s) \\
\end{aligned}
\label{eq:sample}
\end{equation}

Special care must be taken here: $x_c(t)$ is a continuous signal having the same value 
with the sampled sequence $x_d(n) = x(nT_s)$ at sample instants. So only Fourier 
transform can be exercised here for $x_c(t)$, as shown in equation \ref{eq:dtftlema}.

\begin{equation}
\begin{aligned}
& FT\{x_c(t)\}  =\int_{-\infty}^{\infty}\sum_{n=-\infty}^{\infty}\delta (t-nT_s)x(nT_s)e^{-j\omega t} dt \\
& = \sum_{n=-\infty}^{\infty}\int_{n=-\infty}^{\infty} \delta (t-nT_s)x(nT_s)e^{-j\omega t} dt \\
& = \sum_{n=-\infty}^{\infty}x(nT_s)e^{-j\omega T_s n}\\
 & = \sum_{n=-\infty}^{\infty}x(n)e^{-j\omega_d n}\\
\end{aligned}
\label{eq:dtftlema}
\end{equation}
where $\omega_d = \omega T_s$ is the digital angular frequency in radians per sample,
and $x(n)$ is the discrete sequence of sampled $x(t)$, as introduced before. 

Therefore we define that the Fourier transform of the sampled signal to be 
discrete time Fourier transform, A.K.A. DTFT. This can be described by equation below.

\begin{equation}
FT\{x_c(t)\} = X_c(\omega) = X_d(\omega) = DTFT\{x_d(n)\}
\end{equation}

But, what is the relationship between the spectral of the sample sequence and 
that of the continuous signal? Again we use Fourier transform to make the analysis.

\begin{equation}
\begin{aligned}
& x(nT_s)  = \frac{1}{2\pi}\int_{-\infty}^{\infty}X(\omega)e^{j\omega n T_s} d\omega \\
& = \frac{1}{2\pi}\sum_{k=-\infty}^{\infty}\int_{\frac{k2\pi}{T_s}}^{\frac{(k+1)2\pi}{T_s}}X(\omega)e^{j\omega n T_s} d\omega \\
& \text{assume} \quad \omega  = \omega ' + \frac{2\pi k}{T_s}\\
& = \frac{1}{2\pi}\sum_{k=-\infty}^{\infty}\int_{0}^{\frac{2\pi}{T_s}}
X(\omega' +\frac{2\pi k}{T_s})e^{j(\omega' + \frac{2\pi k}{T_s}) n T_s} d\omega' \\
& = \frac{T_s}{2\pi}\int_{0}^{\frac{2\pi}{T_s}}\frac{1}{T_s}\sum_{k=-\infty}^{\infty}
X(\omega' +\frac{2\pi k}{T_s})e^{j\omega'  n T_s} d\omega' \\
& = DTFT^{-1}\{\frac{1}{T_s}\sum_{k=-\infty}^{\infty}X(\omega' +\frac{2\pi k}{T_s})\}\\
\end{aligned}
\label{eq:xdx}
\end{equation}

Then we can conclude that the relationship between the continuous signal and its sample 
sequence is much like the process of down sampling, as described in equation \ref{eq:cdrelation}.
\begin{equation}
X_c(\omega) = X_d(\omega) = \frac{1}{T_s} \sum_{k=-\infty}^{\infty}X(\omega+\frac{2\pi k}{T_s})
\label{eq:cdrelation}
\end{equation}

Here comes the Nyquist sampling theorem,
\begin{theorem}
if $|X(\omega)|=0$ for all $|\omega| \ge \pi/T_s = \omega_s/2$(Nyquist frequency),
then $X_d(\omega) = 1/T_s X(\omega)$ for $|\omega| < \omega_s/2$, and all 
information in $x(t)$ is preserved in $x_d(n)$. 
\end{theorem}

\section*{Reconstruction}

A practical implementation of D/A conversion is the zero-order hold reconstruction, 
which is piecewise constant within each sample, i.e. $y_r(t) = y_d(t)$, where 
$nT_s  \le t < (n+1)T_s$. Zero-order hold reconstruction can be modeled as 
filtering, which can be described as:
\begin{equation}
\begin{aligned}
& y_c(t)  = \sum_{n=-\infty}^{\infty}y_d(n)\delta(t-nT_s)\\
& h_{zoh}(t)  = 1 \quad  \text{when} 0 \le t < T_s \\
& h_{zoh}(t)  = 0 \quad  \text{otherwise}\\
\end{aligned}
\end{equation}

Then the Fourier transform of the convolution can be calculated in frequency 
domain as multiplication.
\begin{equation}
  \label{eq:zohconv}
  \begin{aligned}
   &  Y_r(\omega)  = H_{zoh}(\omega)Y_c(\omega) \\
   & H_{zoh}(\omega) = FT\{h_{zoh}(t)\}=T_s e^{\frac{-j\omega T_s}{2}}\frac{sin(\frac{\omega T_s}{2})}{\frac{\omega T_s}{2}}\\
   & H_{zoh}(0)  = T_s \\
   & H_{zoh}(\frac{2\pi}{T_s})  = 0 \\
  \end{aligned}
\end{equation}

To sum up, for the entire DSP system, the reconstructed output can be calculated in equation \ref{eq:total}.
Note that $P(\omega)$ and $X_d(\omega)$ are $2\pi/T_s$ periodic. 
\begin{equation}
  \label{eq:total}
  \begin{aligned}
    & Y_r(\omega)  = H_{zoh}(\omega)P(\omega)[\frac{1}{T_s}\sum_{k=-\infty}^{\infty}X(\omega+\frac{2\pi k}{T_s})] \\
    & \text{where}\quad P(\omega) \quad \text{is the DTFT of the processing units} \\
    & \text{and} \quad X_d(\omega) = \frac{1}{T_s}\sum_{k=-\infty}^{\infty}X(\omega+\frac{2\pi k}{T_s})\\
  \end{aligned}
\end{equation}

Sidelobes are caused by ZOH. Since $H_{zoh}(\omega)$ is non-zero outside $[-\pi/T_s \quad \pi/T_s]$, 
low-pass filter is need to eliminate this effect. In the sampling, aliasing is avoided or reduced by 
a low pass filter before the sampling A/D converter. 

\section*{Frequency analysis of signals}

We have already familiar with the convolution-multiplication transformation pair, as shown below.
\begin{equation}
  \label{eq:pair}
  \begin{aligned}
    & y(n) = \sum_{k=-\infty}^{\infty}h(k)x(n-k) \iff Y(\omega) = H(\omega)X(\omega)\\
    & y(n) = x(n)w(n) \iff Y(\omega) = \frac{T_s}{2\pi}\int_{0}^{2\pi f_s}X(\lambda)W(\omega-\lambda)d\lambda\\
  \end{aligned}
\end{equation}
For practical cases, we always focus on certain interval of a infinite long sample sequence, 
and this is known as windowing. Hence the signal we actually interest in is the multiplication
of the ideal infinite long sample sequence and the window functions. For example, the
rectangular window function can be written as:
\begin{equation}
  \label{eq:rectan}
  \begin{aligned}
&    r_w(n) = 1 \quad \text{when} \quad n = 0 \dots N-1 \\
&    r_w(n) = 0 \quad \text{otherwise}\\
  \end{aligned}
\end{equation}

Let $\hat{x}(n) = r_w(n)x(n)$, and $R_w(\omega)=\sum_{n=-\infty}^{\infty}r_w(n)e^{-j\omega n}
=\sum_{n=0}^{N-1}e^{-j\omega n}=\frac{1-e^{j\omega N}}{1-e^{j\omega}}
=e^{-j(N-2)/2\omega}sin(N\omega /2)/sin(\omega /2)$. We can see that 
$R_w(0) = N$, and $R_w(\omega)=0$ for $\omega = 2\pi/N \quad 4\pi/N \dots $. 
$R_w(\omega)$ is periodic with $2\pi$. The mainlobe width is proportional to 
$1/N$, when $N$ increases to infinity, the Fourier transform of the window 
function converges to Dirac delta function. The peak level ratio between 
mainlobe and sidelobe are constant 13dB. 

Since $\hat{X}(\omega) = 1/2\pi\int X(\omega-\lambda)R_w(\lambda)d\lambda$,
when $\omega \rightarrow \infty$, $R_w(\lambda) = 2\pi\delta(\lambda)$, 
then $\implies \hat{X}(\omega) = X(\omega)$.

If we sample the windowed DTFT at frequencies $\omega_k=2\pi k/N$, 
then we have defined the Discrete Fourier Transform(DFT), shown in equation \ref{eq:DFT}.
\begin{equation}
  \label{eq:DFT}
  \begin{aligned}
    & \hat{X}(\omega)=\sum_{n=0}^{N-1}x(n)e^{-j\omega n} \\
    & X(k) = \hat{\frac{2\pi k}{N}} = \sum_{n=0}^{N-1}x(n)e^{-j\frac{2\pi k}{N}n}\\
    & k = 0 \dots N-1 \\
  \end{aligned}
\end{equation}

The inverse DFT has been defined in equation \ref{eq:IDFT}.
\begin{equation}
  \label{eq:IDFT}
  \begin{aligned}
    & x(n) = \frac{1}{N}\sum_{k=0}^{N-1}X(k)e^{j\frac{2\pi k}{N}n}\\
    & n = 0 \dots N-1 \\
  \end{aligned}
\end{equation}

DFT is linear invertible transformation. $X(N) = X(0)$ because $X(k)$ is 
$N$ periodic function. $x(n)$ is also $N$ periodic if calculated by IDFT.
The IDFT relation is a Fourier series representation of the N-periodic signal. 

