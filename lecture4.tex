\chapter*{Lecture 4}

\section*{Filter Design}

In order to design digital filters, we start by revisiting the linear convolution:
\begin{equation}
  \label{eq:firconv}
  y(n) = \sum_{k=-\infty}^{\infty}h(k)x(n-k)=\sum_{k=-\infty}^{\infty}x(k)h(n-k)
\end{equation}
Infinite length of the filter and non-causality are not suitable for real implementation.
Thus we have classified the filters by two categories both with causal structure 
(impulse response be zero in time domain when $t<0$) and finite filter length:
finite impulse response and infinite impulse response filter. 

FIR filter has the form of equation \ref{eq:firf}.
\begin{equation}
  \label{eq:firf}
  y(n) = \sum_{k=0}^{M}h(k)x(n-k)
\end{equation}
While IIR filter has the form of equation \ref{eq:iirf},
\begin{equation}
  \label{eq:iirf}
  y(n)=-\sum_{k=1}^{M_a}a(k)y(n-k) + \sum_{k=0}^{M_b}b(k)x(n-k)
\end{equation}
which can also be regarded as a recursive filter since it uses the history values 
of output for present output calculation, also known as feedback (loop) structure.  

As can be seen from equations above, IIR filter is a generalization of FIR filter. 
Most filters are designed to be frequency selective, so we design and analyze the 
filter properties in frequency domain. Sometimes however, an alternative class of 
filters are designed based on time domain properties, for example signal matched 
filters.

\section*{The Window Method}
We try to explain this method by example demonstration, where a low pass 
filter is designed. The ideal mathematical model of low pass filter is good starting point. 
An ideal low pass filter can be modeled as a "brick wall" in its frequency function, shown in equation 
\ref{eq:brickwall},
\begin{equation}
  \label{eq:brickwall}
  H(\omega) = \left\{ \begin{array}{cc}
       1  &  |\omega| \leq \omega_k \\
       0  &  \omega_k < |\omega| \leq \pi \\
       \end{array} \right.
\end{equation}
given unity sampling frequency. Hence the inverse DTFT of this frequency 
function would be
\begin{equation}
  \label{eq:bwalltime}
  h(n) = \left\{ \begin{array}{cc}
       \frac{\omega_k}{\pi}  &  n=0 \\
       \frac{sin(\omega_k n)}{\pi n}  &  n \neq 0 \\
       \end{array} \right.
\end{equation}
Note that resulting $h(n)$ is so far non-causal and infinite long, which 
should be modified for physical implementation. 

Consequently, the task of filter design now has been reduced to 
how to select $h(k) \quad k=0 \dots M$ such that the filter approximates 
the ideal behavior. This definitely involves trade-off. Even more, 
the inverse DFT we developed in previous lectures now turns 
out to be an improper technique for filter design because 
it only defines the frequency behavior at limited points. 

The window method for FIR design now becomes finding the 
finite and causal impulse response $H_d(\omega)$ as the desired response,
in equation \ref{eq:firapp}.
\begin{equation}
  \label{eq:firapp}
  h_d(\omega)=\frac{1}{2\pi}\int_{-\pi}^{\pi}H_d(\omega)e^{j\omega n}d\omega \quad n=0, \pm 1, \pm 2 \dots
\end{equation}
Say 
\begin{equation}
  \label{eq:firdesire}
  H_d(\omega) = \left\{
    \begin{array}{cc}
      1 & |\omega| < \omega_c \\
      0 & \omega_c < |\omega| < 2\pi \\
    \end{array}
    \right.
\end{equation} 
we have $h_d(n)$ calculated as 
\begin{equation}
  \label{eq:firidtft}
  \begin{aligned}
    h_d(n) & = \frac{1}{2\pi}\int_{-\omega_c}^{\omega_c}1 (cos(\omega n) + j sin(\omega n)) d\omega \\
             & = \frac{1}{2\pi} \int_{-\omega_c}^{\omega_c}cos(\omega n) d\omega \\
            & = \frac{sin(\omega_c n)}{\pi n}\\
  \end{aligned}
\end{equation}

Then we make a windowed and delayed version of $h_d(n)$, i.e. we (1) truncate the 
impulse response of the desired filter with certain symmetrical interval and (2) then
make a time shift of length $M/2$ to the right, where $M$ is the total length of the 
filter response.

(1) Truncation. 
\begin{equation}
  \label{eq:truncation}
  h_T(n) = w(n)h_d(n) = \left\{
    \begin{array}{cc}
      w(n)h_d(n) & n = 0,\pm 1 \dots \pm M/2 \\
      0 & otherwise \\
    \end{array}
     \right.
\end{equation}
This indicates that the truncated filter has the frequency behavior characterized 
as the convolution of the desired filter and the window function in frequency domain.
\begin{equation}
  \label{eq:firwin}
  H_T(\omega) = \frac{1}{2\pi}\int_{-\pi}^{\pi}H_d(\lambda)W(\omega-\lambda) d \lambda
\end{equation}
If $w(n)$ is rectangular window, then we have
\begin{equation}
  \label{eq:recwin}
  W(\omega) = \sum_{n=-\frac{M}{2}}^{\frac{M}{2}}1 e^{-j\omega n} = \frac{sin(\omega(N-1)/2)}{sin(\omega /2)}
\end{equation}

(2) Time shift, delay $M/2$.
\begin{equation}
  \label{eq:firdelay}
  h(n)=h_T(n-\frac{M}{2}) \implies H(\omega) = e^{-j\omega M/2} H_T(\omega)
\end{equation}
which introduces a linear phase shift. 

The window function $w(n)$ is a design parameter, which introduces trade-off between main-lobe width
(the width of transition region); the side-lobe height is related to stop-band attenuation. 

\begin{table}[H]
\begin{center}
\caption{Properties of several window functions}
\scalebox{1.0}{
\begin{tabular}{|c|c|c|c|c|}
\hline
window function & main-lobe width & side-lobe & transition band & stop band rejection \\
\cline{1-5}
  rectangular & $4\pi / M$ & -13dB & $1/N$ & 21dB\\
\cline{1-5}
  Hamming & $8\pi /M$  & -31dB & $33/N$& 53dB\\
\cline{1-5}
Blackman & $12\pi /M$ & -57dB & $5.5/N$ & 74dB \\
\hline
\end{tabular}
}
\label{winproperty}
\end{center}
\end{table} 

\section*{Symmetric FIR filters}
\begin{equation}
  \label{eq:symm}
  h(n) = h(M-n) \implies H(\omega) = H_r(\omega)e^{-j\omega M/2}
\end{equation}
where $H_r(\omega)$ is a real function. We can see that $H(\omega)$ always has 
a frequency response with linear phase shift. 
\begin{equation}
  \label{eq:linearphase}
  \angle{H(\omega)}= \left\{
    \begin{array}{cc}
      -\omega M/2 & if \quad H_r(\omega) > 0 \\
      -\omega M/2 + \pi & if \quad H_r(\omega) < 0 \\
    \end{array}
    \right.
\end{equation}
Note that there might be break points in phase plots when using matlab function "angle".