
\chapter*{Lecture 1}

\section*{Introduction}

Signals are everywhere. For example voltage, current, electromagnetic fields, acoustic signals, sound waves, 
vibrations, angular motions, velocities, they are all continuous in time. Some signals are continuous in space,
like sound pressure. 

Sampling leads to the ability to process the signals algorithmically perhaps in a computer. The availability of low 
cost low power digital devices has lead to an enormous development of DSP application:
 speech, audio, music, noise reduction, coding compression(MPEG, MP3, CD, DVD);
 image, movie;
 radio, mobile phones, modulations, transmission;
 radar, sonar, target detection, tracking;
 control, servo, mechanics, plant control;
 biomedical, analysis diagnosis, monitoring, telemedicine;

But what on earth is a DSP system? Basically it covers the following functions:
processing of signals : filtering, modulation, etc, and these are characterized as change of the signals;
analysis of signals : transform, model based analysis, etc, and these are characterized as extracting features;
the last category is detection and classification, and we will not discuss this topic in this course.

In DSP implementation, the basic building block is the gate logic, this is almost the lowest level of the system. 
There are other different layers of abstraction. Desktop computer can be put at the higest level, 
where the development relies purely on software. Dedicated signal processor, which is processor$+$software style,
is in the middle. ASIC/FPGA is at the bottom. 

A DSP system has certain properties. In a DSP system, signal is always quantized(in amplitude) and 
sampled(in temporal), which leads to not exact response of its continuous counterpart. 
AD/DA converters impose extra costs to the system. 
Besides the system has limited bandwidth, because of limited clock frequencies of digital circuits.
Although at all those costs, a DSP system has many attracting features. It has a good control of accuracy, 
no drift-away temperature affections that usually encountered in analog circuits. 
Many complex algorithms can be realized. Flexibility and adaptivity can also be achieved. 
Another interesting pro is that low frequency is easy to achieve, which is hard for analog circuits.

\section*{Fourier Transfrom}

signal processing classically deals with spectral contents. Why? 
\textbf{The complex exponential $e^{j\omega t}$ forms a basis for all solutions to linear dynamic systems}.
The Fourier or Laplace transform forms the analytical means of such solutions.
\begin{equation}
\begin{aligned}
 &{X}(\omega) = \mathcal{F}[x(t)] \triangleq \int_{-\infty}^{\infty} x(t)e^{-j\omega t} dt \\
 & x(t) = \mathcal{F}^{-1}[{X}(\omega)] \triangleq \frac{1}{2\pi} \int_{-\infty}^{\infty} x(t)e^{j\omega t} d\omega \\
\end{aligned}
\end{equation}

The signal energy can be written as
\begin{equation}
{E} = \int_{-\infty}^{\infty} |x(t)|^2dt = \frac{1}{2\pi}\int_{-\infty}^{\infty}|{X}(\omega)|^2d\omega
\end{equation}
where $|{X}(\omega)|$ as the amplitude of Fourier transform of $x(t)$ shows how the signal can be 
decomposed into different frequency components. Similarly, $|{X}(\omega)|^2$ shows how the power 
is distributed along the frequency axis. Here we introduce Dirac delta function for analog signals. 
\begin{equation}
\begin{aligned}
&\delta(t)  = 0, t \neq 0 \\
&\delta(t)  = \infty, t = 0 \\
&\int_{-\infty}^{\infty}\delta(t) dt = 1\\ 
&\int_{-\infty}^{\infty}\delta(t-a)f(t) dt  = f(a)\\ 
\end{aligned}
\end{equation}

These are some properties for Fourier transform.

\begin{table}[H]
\begin{center}
\caption{Properties of Fourier Transform}
\scalebox{1.0}{
\begin{tabular}{|c|c|c|}
\hline
     \multicolumn{2}{|c|}{$x(t)$}   &  ${X}(\omega)$ \\ 
\cline{1-3}
linearity  & $ax(t) + by(t)$  &  $a{X}(\omega) + b{Y}(\omega)$ \\
\cline{1-3}
swapping  &  ${X}(t)$  & $2\pi x(-\omega)$ \\
\cline{1-3}
scale in time domain & $x(at)$  &  $\frac{1}{|a|}{X}(\frac{\omega}{a})$ \\
\cline{1-3}
 delay-modulation dual &$x(t-t_0)$  &  ${X}(\omega)e^{-jt_0\omega}$  \\ 
\cline{1-3}
 delay-modulation dual &  $e^{j\omega_0t}x(t)$ & ${X}(\omega - \omega_0)$ \\
\cline{1-3}
  differentiate dual & $\frac{d^n x(t)}{dt^n}$ & $(j\omega)^n{X}(\omega)$ \\
\cline{1-3}
 differentiate dual & $t^n x(t)$ & $j^n \frac{d^n{X}(\omega)}{d\omega^n}$ \\
\cline{1-3}
multiplication-convolution dual &  $\int_{-\infty}^{\infty}h(\tau)x(t-\tau)d\tau$   &   ${H}(\omega){X}(\omega)$  \\
\cline{1-3}
multiplication-convolution dual & $h(t)x(t)$ & $\frac{1}{2\pi} \int_{-\infty}^{\infty}{H}(\tau){X}(\omega-\tau) d\tau$ \\
\cline{1-3}
   constant(infinite energy)& $1$ & $2\pi\delta(\omega)$ \\
\cline{1-3}
  delta function  &  $\delta(t)$  &  1 \\
\hline
\end{tabular}
}
\label{fourierproperty}
\end{center}
\end{table} 


